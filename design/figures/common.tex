\documentclass{standalone}

\usepackage{luatex85}
\renewcommand{\familydefault}{\sfdefault}
\input figures/figures.sty

\begin{document}

    \begin{tikzpicture}[scale=\scale, every node/.style={scale=\scale}]

        \begin{scope}
            \begin{abstractclass}[text width=\smallTextWidth]{HeteroCouples}{0, 0}
                \operation[0]{\procedure{construct}{numComponents: \type{int}}}
                \operation{\procedure{destroy}{}}
                \operation{\procedure{getNumIndices}{}: \type{int}}
                \operation{\procedure{get}{index: \type{int}}: \type{int[2]}}
            \end{abstractclass}
        \end{scope}

        \begin{scope}[shift={($(HeteroCouples.south) - (0 cm, 2 cm)$)}]
            \begin{class}[text width=\smallTextWidth]{PotentialDomain}{0, 0}
                \UMLattribute{min: \type{real}}
                \UMLattribute{delta: \type{real}}
            \end{class}
        \end{scope}

        \begin{scope}[shift={($(PotentialDomain.south) + (-4 cm, -2 cm)$)}]
            \begin{class}[text width=\smallTextWidth]{ShortPotentialDomain}{0, 0}
                \inherit{PotentialDomain}
                \UMLattribute{max: \type{real}}
            \end{class}
        \end{scope}

        \begin{scope}[shift={($(PotentialDomain.south) + (4 cm, -2 cm)$)}]
            \begin{class}[text width=\smallTextWidth]{DipolarPotentialDomain}{0, 0}
                \inherit{PotentialDomain}
                \UMLattribute{maxOverBox: \type{real}}
            \end{class}
        \end{scope}

    \end{tikzpicture}

\end{document}
