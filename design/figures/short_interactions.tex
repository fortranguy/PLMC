\documentclass{standalone}

\input figures/figures.sty

\renewcommand{\familydefault}{\sfdefault}
\begin{document}

    \begin{tikzpicture}[scale=\scale, every node/.style={scale=\scale}]

        \begin{scope}
            \begin{abstractclass}[text width=\mediumTextWidth]{VisitableCells}{0, 0}
                \operation{\procedure{construct}{periodicBox: \type{PeriodicBox},
                    positions: \type{ComponentCoordinates}, hardContact: \type{HardContact},
                    pairPotential: \type{PairPotential}, neighbourCells: \type{NeighbourCells},
                    listMold: \type{VisitableList}}}
                \operation{\procedure{destroy}{}}
                \operation{\procedure{reset}{}}
                \operation{\procedure{visitEnergy}{overlap: \type{bool}, energy: \type{real},
                    particle: \type{TemporaryParticle},
                    inRange: \procedure{inRange}{iInside: \type{int}, iOutside: \type{int}},
                    iExclude: \type{int}}}
                \operation{\procedure{visitContacts}{overlap: \type{bool}, contacts: \type{real},
                    particle: \type{TemporaryParticle}}}
                \operation{\procedure{translate}{toPosition: \type{real[3]},
                    from: \type{TemporaryParticle}}}
                \operation{\procedure{add}{particle: \type{TemporaryParticle}}}
                \operation{\procedure{remove}{particle: \type{TemporaryParticle}}}
            \end{abstractclass}
        \end{scope}

        \begin{scope}[shift={($(VisitableCells.south) - (0 cm , 1.5 cm)$)}]
            \begin{abstractclass}[text width=\mediumTextWidth]{NeighbourCells}{0, 0}
                \operation{\procedure{construct}{periodicBox: \type{PeriodicBox},
                    hardContact: \type{HardContact}, pairPotential: \type{PairPotential}}}
                \operation{\procedure{destroy}{}}
                \operation{\procedure{reset}{}}
                \operation{\procedure{getGlobalLbounds}{}: \type{int[3]}}
                \operation{\procedure{getGlobalUbounds}{}: \type{int[3]}}
                \operation{\procedure{get}{localI1: \type{int}, localI2: \type{int},
                    localI3: \type{int}, globalI1: \type{int}, globalI2: \type{int},
                    globalI3: \type{int}}: \type{int[3]}}
                \operation{\procedure{skip}{atBottomLayer: \type{bool},
                    atTopLayer: \type{bool}, localI3: \type{int}}: \type{bool}}
                \operation{\procedure{index}{position: \type{real[3]}}: \type{int[3]}}
            \end{abstractclass}
        \end{scope}

        \begin{scope}[shift={($(VisitableCells.north) - (9 cm, 0 cm)$)}]
            \begin{abstractclass}[text width=\mediumTextWidth]{VisitableList}{0, 0}
                \operation{\procedure{construct}{periodicBox: \type{PeriodicBox},
                    positions: \type{ComponentCoordinates}, hardContact: \type{HardContact}}}
                \operation{\procedure{destroy}{}}
                \operation{\procedure{set}{iTarger: \type{int}, iParticle: \type{int}}}
                \operation{\procedure{visitEnergy}{overlap: \type{bool}, energy: \type{real},
                    particle: \type{TemporaryParticle},
                    pairPotential: \type{PairPotential},
                    inRange: \procedure{inRange}{iInside: \type{int}, iOutside: \type{int}},
                    iExclude: \type{int}}}
                \operation{\procedure{visitContacts}{overlap: \type{bool}, contacts: \type{real},
                    particle: \type{TemporaryParticle},
                    pairPotential: \type{PairPotential}}}
                \operation{\procedure{add}{iParticle: \type{int}}}
                \operation{\procedure{remove}{iParticle: \type{int}}}
            \end{abstractclass}
        \end{scope}

        \begin{scope}[shift={($(VisitableCells.north) + (0 cm, 5 cm)$)}]
            \begin{abstractclass}[text width=\mediumTextWidth]{HardContact}{0, 0}
                \operation{\procedure{construct}{halfDistribution: \type{HalfDistribution}}}
                \operation{\procedure{destroy}{}}
                \operation{\procedure{getMaxDistance}{}: \type{real}}
                \operation{\procedure{meet}{overlap: \type{bool}, contact: \type{real},
                    minDistance: \type{real}, vector: \type{real[3]}}}
            \end{abstractclass}
        \end{scope}

        \begin{scope}[shift={($(VisitableList.north |- HardContact.north)$)}]
            \begin{abstractclass}[text width=\smallTextWidth]{HalfDistribution}{0, 0}
                \operation[0]{\procedure{set}{\ldots}}
                \operation{\procedure{getMaxDistance}{}: \type{real}}
                \operation{\procedure{get}{distance: \type{real}}: \type{real}}
            \end{abstractclass}
        \end{scope}

        \begin{scope}[shift={($(VisitableList.south) - (0 cm, 1.5 cm)$)}]
            \begin{abstractclass}[text width=\mediumTextWidth]{PairPotential}{0, 0}
                \operation[0]{\procedure{construct}{domain: \type{ShortPotentialDomain},
                    expression: \type{PotentialExpression}}}
                \operation[0]{\procedure{destroy}{}}
                \operation{\procedure{getMinDistance}{}: \type{real}}
                \operation{\procedure{getMaxDistance}{}: \type{real}}
                \operation[0]{\procedure{meet}{overlap: \type{bool}, energy: \type{real},
                    distance: \type{real}}}
            \end{abstractclass}
        \end{scope}

        \begin{scope}[shift={($(PairPotential.south) - (0 cm, 1.5 cm)$)}]
        \begin{abstractclass}[text width=\smallTextWidth]{PotentialExpression}{0, 0}
                \operation{\procedure{get}{distance: \type{real}}: \type{real}}
            \end{abstractclass}
        \end{scope}

        \begin{scope}[shift={($(VisitableCells.north) + (9 cm, 0 cm)$)}]
            \begin{class}[text width=\mediumTextWidth]{ShortInteractions}{0, 0}
                \attribute{wallsVisitor: \type{WallsVisitor}}
                \attribute{wallPairs: \type{PairPotential[:]}}
                \attribute{componentsVisitor: \type{ShortPairsVisitor}}
                \attribute{componentsPairs: \type{PairPotential[:.]}}
                \attribute{neighbourCells: \type{NeighbourCells[:.]}}
                \attribute{visitableCells: \type{VisitableCells[::]}}
            \end{class}
        \end{scope}

        \begin{scope}[shift={($(ShortInteractions.south) - (0 cm, 1.5 cm)$)}]
            \begin{abstractclass}[text width=\mediumTextWidth]{ShortPairsVisitor}{0, 0}
                \operation{\procedure{construct}{periodicBox: \type{PeriodicBox}}}
                \operation{\procedure{destroy}{}}
                \operation{\procedure{visitInter}{overlap: \type{bool}, energy: \type{real},
                    positions1: \type{ComponentCoordinates},
                    positions2: \type{ComponentCoordinates},
                    pairPotential: \type{PairPotential}}}
                \operation{\procedure{visitIntra}{overlap: \type{bool}, energy: \type{real},
                    positions: \type{ComponentCoordinates}, pairPotential: \type{PairPotential}}}
            \end{abstractclass}
        \end{scope}

        \begin{scope}[shift={($(HardContact.north -| ShortInteractions)$)}]
            \umlnote[text width=\mediumTextWidth] {\import{}: \\
                \namespace{Environment} :: PeriodicBox, WallsVisitor \\
                \namespace{Mixture} :: ComponentCoordinates, TemporaryParticle};
        \end{scope}

        \begin{scope}[on background layer]
            \aggregation{VisitableCells}{\refName{neighbourCells}}{\numRef{1}}{NeighbourCells}
            \aggregation{VisitableCells}{\refName{pairPotential}}{\numRef{1}}{PairPotential}
            \composition{VisitableCells}{\refName{visitableList}}{\numRef{1..n}}{VisitableList}
            \composition{PairPotential}{\refName{potentialExpression}}{\numRef{1}}
                {PotentialExpression}
            \aggregation{VisitableList}{\refName{hardContact}}{\numRef{1}}{HardContact}
            \composition{HardContact}{\refName{halfDistribution}}{\numRef{1}}{HalfDistribution}
        \end{scope}

    \end{tikzpicture}

\end{document}
