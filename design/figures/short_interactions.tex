\documentclass{standalone}

\input figures/figures.sty

\renewcommand{\familydefault}{\sfdefault}
\begin{document}

    \begin{tikzpicture}[scale=\scale, every node/.style={scale=\scale}]

        \begin{scope}
            \begin{abstractclass}[text width=\mediumTextWidth]{PairPotential}{0, 0}
                \operation[0]{\procedure{construct}{domain\type{ShortPotentialDomain},
                    expression\type{PotentialExpression}}}
                \operation[0]{\procedure{destroy}{}}
                \operation{\procedure{getMaxDistance}{}\type{real}}
                \operation[0]{\procedure{meet}{overlap\type{bool}, energy\type{real},
                    distance\type{real}}}
            \end{abstractclass}
        \end{scope}

        \begin{scope}[shift={($(PairPotential.north) + (8 cm, 0 cm)$)}]
        \begin{abstractclass}[text width=\smallTextWidth]{PotentialExpression}{0, 0}
                \operation{\procedure{get}{distance\type{real}}\type{real}}
            \end{abstractclass}
        \end{scope}

        \begin{scope}[shift={($(PairPotential.north) + (0 cm, 6.5 cm)$)}]
            \begin{abstractclass}[text width=\mediumTextWidth]{ShortPairsVisitor}{0, 0}
                \operation{\procedure{construct}{periodicBox\type{PeriodicBox}}}
                \operation{\procedure{destroy}{}}
                \operation{\procedure{visitInter}{overlap\type{bool}, energy\type{real},
                    positions1\type{ComponentCoordinates},
                    positions2\type{ComponentCoordinates},
                    pairPotential\type{PairPotential}}}
                \operation{\procedure{visitIntra}{overlap\type{bool}, energy\type{real},
                    positions\type{ComponentCoordinates}, pairPotential\type{PairPotential}}}
            \end{abstractclass}
        \end{scope}

        \begin{scope}[shift={($(PairPotential.north) - (9 cm, 0 cm)$)}]
            \begin{abstractclass}[text width=\mediumTextWidth]{VisitableList}{0, 0}
                \operation{\procedure{construct}{periodicBox\type{PeriodicBox},
                    componentPositions\type{ComponentCoordinates}}}
                \operation{\procedure{destroy}{}}
                \operation{\procedure{set}{iTarger\type{int}, iParticle\type{int}}}
                \operation{\procedure{visit}{overlap\type{bool}, energy\type{real},
                    particle\type{TemporaryParticle},
                    pairPotential\type{PairPotential}, iExclude\type{int}}}
                \operation{\procedure{add}{iParticle\type{int}}}
                \operation{\procedure{remove}{iParticle\type{int}}}
            \end{abstractclass}
        \end{scope}

        \begin{scope}[shift={($(ShortPairsVisitor.north -| VisitableList)$)}]
            \begin{abstractclass}[text width=\mediumTextWidth]{VisitableCells}{0, 0}
                \operation{\procedure{construct}{periodicBox\type{PeriodicBox},
                    positions\type{ComponentCoordinates}, pairPotential\type{PairPotential},
                    listMold\type{VisitableList}}}
                \operation{\procedure{destroy}{}}
                \operation{\procedure{visit}{overlap\type{bool}, energy\type{real},
                    particle\type{TemporaryParticle}, iExclude\type{int}}}
                \operation{\procedure{move}{toPosition\type{double[3]},
                    from\type{TemporaryParticle}}}
                \operation{\procedure{add}{particle\type{TemporaryParticle}}}
                \operation{\procedure{remove}{particle\type{TemporaryParticle}}}
            \end{abstractclass}
        \end{scope}

        \begin{scope}[shift={($(ShortPairsVisitor.north -| PotentialExpression.north)$)}]
            \umlnote[text width=\smallTextWidth] {\import{}: \\
                \namespace{Environment} :: PeriodicBox \\
                \namespace{Mixture} :: ComponentCoordinates, TemporaryParticle};
        \end{scope}

        \begin{scope}[on background layer]
            \aggregation{VisitableCells}{\refName{pairPotential}}{\numRef{1}}{PairPotential}
            \composition{VisitableCells}{\refName{visitableList}}{\numRef{1..n}}{VisitableList}
            \composition{PairPotential}{\refName{potentialExpression}}{\numRef{1}}{PotentialExpression}
        \end{scope}

    \end{tikzpicture}

\end{document}
