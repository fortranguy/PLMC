\documentclass{standalone}

\input figures/figures.sty

\renewcommand{\familydefault}{\sfdefault}
\begin{document}

    \begin{tikzpicture}

        \begin{scope}
            \begin{abstractclass}[text width=\smallTextWidth]{Coordinates}{0, 0}
                \operation[0]{\procedure{getNum}{}\type{int}}
                \operation[0]{\procedure{get}{iParticle\type{int}}\type{double[3]}}
            \end{abstractclass}
        \end{scope}

        \begin{scope}[shift={($(Coordinates.south) + (0 cm, -1 cm)$)}]
            \begin{abstractclass}[text width=\mediumTextWidth]{ComponentCoordinates}{0, 0}
                \inherit{Coordinates}
                \operation[0]{\procedure{destroy}{}}
                \operation{\procedure{getNum}{}\type{int}}
                \operation[0]{\procedure{set}{iParticle\type{int}, vector\type{double[3]}}}
                \operation{\procedure{get}{iParticle\type{int}}\type{double[3]}}
                \operation{\procedure{add}{vector\type{double[3]}}}
                \operation{\procedure{remove}{iParticle\type{int}}}
            \end{abstractclass}
        \end{scope}

        \begin{scope}[shift={($(ComponentCoordinates.north) - (9 cm, 0 cm)$)}]
            \begin{abstractclass}[text width=\mediumTextWidth]{ComponentDipolarMoments}{0, 0}
                \inherit{Coordinates}
                \operation{\procedure{construct}{norm\type{double},
                    orientations\type{ComponentCoordinates}}}
                \operation{\procedure{destroy}{}}
                \operation{\procedure{getNum}{}\type{int}}
                \operation{\procedure{getNorm}{}\type{double}}
                \operation{\procedure{get}{iParticle\type{int}}\type{double[3]}}
            \end{abstractclass}
        \end{scope}

        \begin{scope}[shift={($(ComponentCoordinates.south) - (0 cm, 1 cm)$)}]
            \begin{abstractclass}[text width=\smallTextWidth]{ComponentNumber}{0, 0}
                \operation{\procedure{set}{number\type{int}}}
                \operation{\procedure{get}{}\type{int}}
            \end{abstractclass}
        \end{scope}

        \begin{scope}[shift={($(ComponentNumber.north -| ComponentDipolarMoments.south)$)}]
            \begin{abstractclass}[text width=\mediumTextWidth]{ComponentChemicalPotential}{0, 0}
                \operation{\procedure{set}{density\type{double}, excess\type{double}}}
                \operation{\procedure{getDensity}{}\type{double}}
                \operation{\procedure{getExcess}{}\type{double}}
            \end{abstractclass}
        \end{scope}

        \begin{scope}[shift={($(ComponentCoordinates.north) + (9 cm, 0 cm)$)}]
            \begin{class}[text width=\mediumTextWidth]{Component}{0, 0}
                \attribute{number\type{ComponentNumber}}
                \attribute{positions\type{ComponentCoordinates}}
                \attribute{orientations\type{ComponentCoordinates}}
                \attribute{chemicalPotential\type{ComponentChemicalPotential}}
                \attribute{dipolarMoments\type{ComponentDipolarMoments}}
            \end{class}
        \end{scope}

        \begin{scope}[shift={($(Component.south) - (0 cm, 1 cm)$)}]
            \begin{abstractclass}[text width=\smallTextWidth]{MinimumDistance}{0, 0}
                \operation{\procedure{set}{minDistance\type{double}}}
                \operation{\procedure{get}{}\type{double}}
            \end{abstractclass}
        \end{scope}

        \begin{scope}[shift={($(Component.north) + (0 cm, 6.5 cm)$)}]
            \begin{abstractclass}[text width=\mediumTextWidth]{MixtureTotalMoment}{0, 0}
                \operation{\procedure{construct}{components\type{Component[:]},
                    areDipolar\type{bool[:]}}}
                \operation{\procedure{destroy}{}}
                \operation{\procedure{reset}{}}
                \operation{\procedure{isDipolar}{iComponent\type{int}} }
                \operation{\procedure{get}{}}
                \operation{\procedure{add}{iComponent\type{int}, dipolarMoment\type{double[3]}}}
                \operation{\procedure{remove}{iComponent\type{int}, dipolarMoment\type{double[3]}}}
            \end{abstractclass}
        \end{scope}

        \begin{scope}[on background layer]
            \aggregation{MixtureTotalMoment}{\refName{components}}{\numRef{1..n}}{Component}
        \end{scope}

    \end{tikzpicture}

\end{document}
\renewcommand{\familydefault}{\rmdefault}
