\documentclass{standalone}

\usepackage{luatex85}
\renewcommand{\familydefault}{\sfdefault}
\input figures/figures.sty

\begin{document}

    \begin{tikzpicture}[scale=\scale, every node/.style={scale=\scale}]

        \begin{scope}
            \begin{class}[text width=\mediumTextWidth]{ChangesComponent}{0, 0}
                \UMLattribute{translatedPositions: \type{MovedCoordinates}}
                \UMLattribute{rotatedOrientations: \type{MovedCoordinates}}
                \UMLattribute{translationTuner: \type{MoveTuner}}
                \UMLattribute{rotationTuner: \type{MoveTuner}}
            \end{class}
        \end{scope}

        \begin{scope}[shift={($(ChangesComponent.south) - (0 cm, 1.5 cm)$)}]
            \begin{abstractclass}[text width=\mediumTextWidth]{MoveTuner}{0, 0}
                \operation{\procedure{construct}{movedCoordinates: \type{MovedCoordinates},
                    parameters: \type{MoveTunerParameters}}}
                \operation{\procedure{destroy}{}}
                \operation{\procedure{tune}{tuned: \type{bool}, iStep: \type{int},
                    successRatio: \type{real}}}
            \end{abstractclass}
        \end{scope}

        \begin{scope}[shift={($(MoveTuner.north) - (7.5 cm, 0 cm)$)}]
            \begin{abstractclass}[text width=\smallTextWidth]{MovedCoordinates}{0, 0}
                \operation[0]{\procedure{construct}{\ldots}}
                \operation[0]{\procedure{destroy}{}}
                \operation[0]{\procedure{increaseDelta}{}}
                \operation[0]{\procedure{decreaseDelta}{}}
            \end{abstractclass}
        \end{scope}

        \begin{scope}[shift={($(MoveTuner.south) - (0 cm, 1.5 cm)$)}]
            \begin{class}[text width=\smallTextWidth]{MoveTunerParameters}{0, 0}
                \UMLattribute{accumulationPeriod: \type{int}}
                \UMLattribute{wantedSuccessRatio: \type{real}}
                \UMLattribute{tolerance: \type{real}}
            \end{class}
        \end{scope}

        \begin{scope}[shift={($(MovedCoordinates |- MoveTunerParameters.north)$)}]
            \begin{abstractclass}[text width=\smallTextWidth]{MovedComponentCoordinates}{0, 0}
                \inherit{MovedCoordinates}
                \operation[0]{\procedure{get}{iParticle: \type{int}}: \type{real[3]}}
            \end{abstractclass}
        \end{scope}

        \begin{scope}[shift={($(MovedCoordinates |- ChangesComponent.north)$)}]
            \begin{abstractclass}[text width=\smallTextWidth]{ChangedBoxSize}{0, 0}
                \inherit{MovedCoordinates}
                \operation{\procedure{getRatio}{}: \type{real[3]}}
            \end{abstractclass}
        \end{scope}

        \begin{scope}[shift={($(MoveTunerParameters.north) + (8.5 cm, 0 cm)$)}]
            \begin{abstractclass}[text width=\mediumTextWidth]{CoordinatesCopier}{0, 0}
                \operation[0]{\procedure{construct}{\ldots}}
                \operation{\procedure{destroy}{}}
                \operation{\procedure{copy}{target: \type{real[3]}, source: \type{real[3]},
                    ijComponents: \type{int[2]}}}
            \end{abstractclass}
        \end{scope}

        \begin{scope}[shift={($(MoveTuner.north -| CoordinatesCopier)$)}]
            \begin{class}[text width=\mediumTextWidth]{Changes}{0, 0}
                \UMLattribute{changedBoxSize: \type{ChangedBoxSize}}
                \UMLattribute{boxSizeChangeTuner: \type{MoveTuner}}
                \UMLattribute{randomPosition: \type{RandomCoordinates}}
                \UMLattribute{randomOrientation: \type{RandomCoordinates}}
                \UMLattribute{positionCopier: \type{CoordinatesCopier}}
                \UMLattribute{orientationCopier: \type{CoordinatesCopier}}
                \UMLattribute{components: \type{ChangesComponent[:]}}
            \end{class}
        \end{scope}

        \begin{scope}[shift={($(ChangesComponent.north -| Changes.north)$)}]
            \umlnote[text width=\mediumTextWidth]{\import{}:\\
                \namespace{Random} :: RandomCoordinates};
        \end{scope}

        \begin{scope}[on background layer]
            \aggregation{MoveTuner}{\refName{movedCoordinates}}{\numRef{1}}{MovedCoordinates}
        \end{scope}

    \end{tikzpicture}

\end{document}
