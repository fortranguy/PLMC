\documentclass{standalone}

\input figures/figures.sty

\renewcommand{\familydefault}{\sfdefault}
\begin{document}

    \begin{tikzpicture}[scale=\scale, every node/.style={scale=\scale}]

        \begin{scope}
            \begin{abstractclass}[text width=\mediumTextWidth]{PeriodicBox}{0, 0}
                \operation{\procedure{set}{size: \type{real[3]}}}
                \operation{\procedure{getSize}{}: \type{real[3]}}
                \operation{\procedure{distance}{position1: \type{real[3]},
                    position2: \type{real[3]}}: \type{real}}
                \operation{\procedure{vector}{position1: \type{real[3]}, position2: \type{real[3]}}:
                    \type{real[3]}}
                \operation[0]{\procedure{folded}{position: \type{real[3]}}: \type{real[3]}}
            \end{abstractclass}
        \end{scope}

        \begin{scope}[shift={($(PeriodicBox.north) + (0 cm, 5 cm)$)}]
            \begin{abstractclass}[text width=\mediumTextWidth]{ParallelepipedDomain}{0, 0}
                \operation{\procedure{construct}{periodicBox: \type{PeriodicBox},
                    origin: \type{real[3]}, size: \type{real[3]}}}
                \operation{\procedure{destroy}{}}
                \operation{\procedure{getOrigin}{}: \type{real[3]}}
                \operation{\procedure{getSize}{}: \type{real[3]}}
                \operation{\procedure{isInside}{position: \type{real[3]}}: \type{bool}}
            \end{abstractclass}
        \end{scope}

        \begin{scope}[shift={($(PeriodicBox.north) - (9 cm, 0 cm)$)}]
            \begin{abstractclass}[text width=\mediumTextWidth]{WallsPotential}{0, 0}
                \operation{\procedure{construct}{periodicBox: \type{PeriodicBox}, gap: \type{real},
                    floorPenetration: \type{FloorPenetration}}}
                \operation{\procedure{destroy}{}}
                \operation{\procedure{getGap}{}: \type{real}}
                \operation{\procedure{visit}{overlap: \type{bool}, energy: \type{real},
                    position: \type{real[3]}, pairPotential: \type{PairPotential}}}
            \end{abstractclass}
        \end{scope}

        \begin{scope}[shift={($(ParallelepipedDomain.north -| WallsPotential)$)}]
            \begin{abstractclass}[text width=\mediumTextWidth]{WallsPotentialVisitor}{0, 0}
                \operation{\procedure{construct}{wallsPotential: \type{WallsPotential}}}
                \operation{\procedure{destroy}{}}
                \operation{\procedure{visit}{overlap: \type{bool}, energy: \type{real},
                    positions: \type{ComponentCoordinates}, pairPotential: \type{PairPotential}}}
            \end{abstractclass}
        \end{scope}

        \begin{scope}[shift={($(WallsPotential.south) - (0 cm, 1.5 cm)$)}]
            \begin{abstractclass}[text width=\mediumTextWidth]{FloorPenetration}{0, 0}
                \operation{\procedure{meet}{overlap: \type{bool},
                    shortestVectorFromFloor: \type{real[3]}, positionFromFloor: \type{real[3]}}}
            \end{abstractclass}
        \end{scope}

        \begin{scope}[shift={($(PeriodicBox.north) + (8 cm, 0 cm)$)}]
            \begin{abstractclass}[text width=\smallTextWidth]{Temperature}{0, 0}
                \operation{\procedure{set}{temperature: \type{real}}}
                \operation{\procedure{get}{}: \type{real}}
            \end{abstractclass}
            % Parallel Tempering?
        \end{scope}

        \begin{scope}[shift={($(Temperature.south) - (0 cm, 1.5 cm)$)}]
            \begin{abstractclass}[text width=\smallTextWidth]{ReciprocalLattice}{0, 0}
                \operation{\procedure{contruct}{periodicBox: \type{PeriodicBox},
                    numbers: \type{int[3]}}}
                \operation{\procedure{destroy}{}}
                \operation{\procedure{getNumbers}{}: \type{int[3]}}
            \end{abstractclass}
        \end{scope}

        \begin{scope}[shift={($(Temperature.north) + (0 cm, 3.5 cm)$)}]
            \begin{abstractclass}[text width=\smallTextWidth]{Permittivity}{0, 0}
                \operation{\procedure{set}{permittivity: \type{real}}}
                \operation{\procedure{get}{}: \type{real}}
            \end{abstractclass}
        \end{scope}

        \begin{scope}[shift={($(Permittivity.north) + (0 cm, 3.5 cm)$)}]
            \begin{abstractclass}[text width=\smallTextWidth]{FieldExpression}{0, 0}
                \operation{\procedure{get}{position: \type{real[3]}}: \type{real[3]}}
            \end{abstractclass}
        \end{scope}

        \begin{scope}[shift={($(ParallelepipedDomain.north) + (0 cm, 4.5 cm)$)}]
            \begin{abstractclass}[text width=\mediumTextWidth]{ExternalField}{0, 0}
                \operation{\procedure{construct}{parallelepipedDomain: \type{ParallelepipedDomain},
                    fieldExpression: \type{FieldExpression}}}
                \operation{\procedure{destroy}{}}
                \operation{\procedure{get}{position: \type{real[3]}}: \type{real[3]}}
            \end{abstractclass}
        \end{scope}

        \begin{scope}[shift={($(ExternalField.north -| WallsPotentialVisitor)$)}]
            \begin{class}[text width=\mediumTextWidth]{Environment}{0, 0}
                \attribute{periodicBox: \type{PeriodicBox}}
                \attribute{temperature: \type{Temperature}}
                \attribute{externalField: \type{ExternalField}}
                \attribute{permittivity: \type{Permittivity}}
                \attribute{reciprocalLattice: \type{ReciprocalLattice}}
                \attribute{wallsPotential: \type{WallsPotential}}
            \end{class}
        \end{scope}

        \begin{scope}[shift={($(ExternalField.north -| FieldExpression.north)$)}]
            \umlnote[text width=\smallTextWidth]{\import{}:\\
                \namespace{ShortInteractions} :: PairPotential};
        \end{scope}

        \begin{scope}[on background layer]
            \aggregation{ParallelepipedDomain}{\refName{periodicBox}}{\numRef{1}}{PeriodicBox}
            \aggregation{ReciprocalLattice}{\refName{periodicBox}}{\numRef{1}}{PeriodicBox}
            \composition{ExternalField}{\refName{parallelepipedDomain}}{\numRef{1}}
                {ParallelepipedDomain}
            \composition{ExternalField}{\refName{fieldExpression}}{\numRef{1}}{FieldExpression}
            \aggregation{WallsPotential}{\refName{periodicBox}}{\numRef{1}}{PeriodicBox}
            \composition{WallsPotential}{\refName{floorPenetration}}{\numRef{1}}{FloorPenetration}
        \end{scope}
    \end{tikzpicture}

\end{document}
