\documentclass{article}

\usepackage[english]{babel}
\usepackage[utf8]{inputenc}
\usepackage[T1]{fontenc}

\usepackage[simplified]{pgf-umlcd}
\usetikzlibrary{calc, backgrounds}

\usepackage[a4paper]{geometry}

\title{Emergence of Design Patterns in the architecture of a Monte-Carlo program
    simulating apolar and dipolar spheres mixture}
\author{Lucas Levrel and Salomon Chung}
\date{February 17, 2015 - \today{}}

\begin{document}

    \maketitle

    \begin{abstract}
        A classical Monte-Carlo simulation of liquids has a simple and well-defined structure:
        1. Intialisation, 2. Iterations of move trial, 3. Results calculation.
        However, some details may vary such as the geometry (e.g. Bulk / Slab),
        the thermodynamic ensemble (e.g. Canonical / Grand canonical / Isobaric)
        or the particles type and interactions (e.g. Hard or Soft Spheres /
        Apolar or Dipolar Spheres).
        This tension between constance and variability may be addressed by
        the Object-Oriented paradigm (e.g. in Fortran, no kidding).
        Following OOP principles, we notice the natural emergence of Design Patterns.
        Those patterns ensure the architecture of the program will be robust and flexible.
    \end{abstract}

    \begin{figure}[htb]
        \centering
        \input class_diagram.tex
        \caption{UML class diagram}
    \end{figure}

\end{document}
