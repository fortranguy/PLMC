\documentclass{article}

\usepackage[english]{babel}
\usepackage[utf8]{inputenc}
\usepackage[T1]{fontenc}

\usepackage[simplified]{pgf-umlcd}
\usetikzlibrary{calc, backgrounds}

\newif\ifexternal
\externalfalse

\ifexternal
\usetikzlibrary{external}
\tikzexternalize[optimize=false, prefix=Figures/, mode=list and make]
\tikzset{external/system call=
    {pdflatex \tikzexternalcheckshellescape -halt-on-error -interaction=batchmode 
    -jobname "\image" "\texsource" && pdftops -eps "\image".pdf}
}
\tikzset{external/figure name = Figure}
\else
\newcommand{\tikzsetnextfilename}[1]{}
\fi

\usepackage[paper=A4,pagesize]{typearea}
\usepackage{geometry}

\usepackage{biblatex}
\bibliography{references}

\title{Emergence of Design Patterns in the architecture of a Monte-Carlo simulation of liquids}
\author{Lucas Levrel and Salomon Chung}
\date{February 17, 2015 - \today{}}

\begin{document}

    \maketitle

    \begin{abstract}
        A classical Monte-Carlo simulation of liquids has a simple and well-defined structure:
        1. Intialisation, 2. Iterations of move trial, 3. Results calculation.
        However, some details may vary such as the geometry (e.g. Bulk / Slab),
        the thermodynamic ensemble (e.g. Canonical / Grand canonical / Isobaric)
        or the particles type and interactions (e.g. Hard or Soft Spheres /
        Apolar or Dipolar Spheres).
        This tension between constance and variability may be addressed by
        the Object-Oriented paradigm (e.g. in Fortran, no kidding).
        By following simple OOP principles, we notice the natural emergence of Design Patterns
        in the architecture.
        Those patterns ensure that the program will be robust and flexible.
    \end{abstract}

    \clearpage
    \KOMAoptions{paper=168cm:237.8cm, pagesize, paper=landscape}
    \recalctypearea
    \newgeometry{top=4cm, bottom=4cm, left=5cm, right=5cm}
    \thispagestyle{empty}
    
    \tikzsetnextfilename{class_diagram}
    \begin{figure}[htb]
        \centering
        \input class_diagram.tex
        \caption{UML class diagram}
    \end{figure}

    \clearpage
    \KOMAoptions{paper=A4, pagesize}
    \restoregeometry
    \recalctypearea

    \nocite{*}
    \printbibliography

\end{document}
