\documentclass{article}

\usepackage{standalone}

\usepackage[english]{babel}
\usepackage[utf8]{inputenc}
\usepackage[T1]{fontenc}

\usepackage[paper=A4,pagesize]{typearea}
\usepackage{geometry}

\usepackage{biblatex}
\bibliography{references}

\title{Application of Object-Oriented Principles and Design Patterns in the architecture of
    a Monte-Carlo simulation of liquids}
\author{Lucas Levrel and Salomon Chung}
\date{February 17, 2015 - \today{}}

\begin{document}

    \maketitle

    \begin{abstract}
        A classical Monte-Carlo simulation of liquids has a simple and well-defined structure:
        1. Intialisation, 2. Iterations of move trial, 3. Results calculation.
        However, some details may vary such as the geometry (e.g. Bulk / Slab),
        the thermodynamic ensemble (e.g. Canonical / Grand canonical / Isobaric)
        or the particles type and interactions (e.g. Hard or Soft Spheres /
        Apolar or Dipolar Spheres).
        This tension between constance and variability may be addressed by
        the Object-Oriented paradigm (e.g. in Fortran, no kidding).
        By following simple OOP principles, we notice the natural emergence of Design Patterns
        in the architecture of our program.
        Those patterns ensure that our simulation program will be robust yet flexible
        to a certain extent.
        
        However, compared to a procedural programming approach, object-oriented programming
        may have more pitfalls.
        For instance, the implementation is likely to be hindered if the overall design is
        not well-defined in advance. Thus an effort must be made upstream.
        
        The notion of \emph{interface} will be fundamental. It will define the general boundaries of
        the program. But it will be the key to switch between local alternatives.
    \end{abstract}

    \clearpage
    \KOMAoptions{paper=168cm:237.8cm, pagesize, paper=landscape}
    \recalctypearea
    \newgeometry{top=4cm, bottom=4cm, left=5cm, right=5cm}
    \thispagestyle{empty}
    
    \begin{figure}[htb]
        \centering
        \includestandalone[mode=buildnew]{figures/class_diagram}
        \caption{UML class diagram}
    \end{figure}

    \clearpage
    \KOMAoptions{paper=A4, pagesize}
    \restoregeometry
    \recalctypearea

    \nocite{*}
    \printbibliography

\end{document}
